
% Fonts
\usepackage[utf8]{inputenc}  % Pour les caractères UTF-8 (nécessaire pour références avec caractères spéciaux)
\usepackage{ebgaramond-maths}
\usepackage[T1]{fontenc}
\usepackage[francais]{babel}

% Hyphenation and text justification
\usepackage{hyphenat}  % Meilleur contrôle de l'hyphenation
\hyphenpenalty=50      % Pénalité pour encourager les coupures de mots (défaut 50)
\exhyphenpenalty=50    % Pénalité pour hyphens explicites
\tolerance=2000        % Tolérance augmentée pour espacements (défaut 200)
\setlength{\emergencystretch}{1.5em}  % Espace d'urgence si justification échoue


% Geometry
\usepackage[left=2cm, right=5cm, top=4cm, bottom=3cm]{geometry}


% Other packages
\usepackage{kantlipsum}
\usepackage{enumitem}
\usepackage[hidelinks]{hyperref}
\usepackage{amsmath, amssymb}
\usepackage{xcolor}
\usepackage[table]{xcolor}
\usepackage{float}
\usepackage{booktabs}  % Pour les tableaux académiques (\toprule, \midrule, \bottomrule)
\usepackage{array}     % Pour les colonnes p{} avec largeur fixe
\usepackage{tikz}      % Pour les diagrammes et schémas
\usetikzlibrary{arrows.meta, positioning, shapes.geometric}
\usepackage{tcolorbox}  % Pour les environnements de définition
\usepackage{pgfplots}   % Pour les graphiques (tailles d'effet)
\pgfplotsset{compat=1.18}
\usepackage[acronym,nomain,toc]{glossaries}  % Pour la liste d'acronymes automatisée
\makeglossaries

% Définir environnement definitionbox minimaliste
\newtcolorbox{definitionbox}[1][]{
  colframe=black,
  coltitle=white,
  colback=white,
  title=#1,
  left=5pt, right=5pt, top=3pt, bottom=3pt,
  boxsep=3pt,
  arc=0pt,
  boxrule=1pt,
  titlerule=0pt
}
