\section*{Résumé}

L'entrée en vigueur du Règlement européen sur l'\gls{ia} (AI Act) en août 2024, classant l'éducation comme secteur à haut risque, impose aux établissements d'enseignement supérieur de structurer leur approche de l'\gls{ia} générative. Ce document de synthèse, destiné à la direction de Polytech Annecy-Chambéry, analyse les fondements techniques et scientifiques de l'\gls{ia} générative, examine le cadre éthique et réglementaire qui encadre son déploiement, documente les pratiques réelles des acteurs de l'\gls{esr}, et propose une feuille de route opérationnelle adaptée aux contraintes d'une école d'ingénieurs.

La méthodologie repose sur une analyse documentaire structurée en sept axes thématiques (fondements, éthique, impacts pédagogiques, usages, gouvernance, \gls{ia} métier, spécificités Polytech), mobilisant plus de 90 sources institutionnelles et scientifiques. Les enquêtes nationales récentes révèlent une adoption massive par les étudiants (75\% dans les Grandes Écoles), modérée par les enseignants-chercheurs (52\%), et limitée par les personnels administratifs, avec des freins spécifiques à chaque public : incertitude réglementaire (74\% des étudiants ne déclarent pas leur usage), manque de formation pratique (65\% des enseignants), et inquiétude sur l'emploi (72\% des BIATSS).

Sept recommandations prioritaires structurent la feuille de route 2026-2028 : coordination formelle avec le référent \gls{ia} de l'\gls{usmb}, enquête interne auprès des trois publics, adoption de la charte \gls{usmb} avec adaptations sectorielles, formation interne mobilisant l'ingénieur de recherche MIAI et les enseignants-chercheurs experts, exploration des options d'infrastructure (ILaaS, \gls{ia} école, partenariat Mistral), identification de champions \gls{ia} par département, et intégration du cadre AIAS pour les évaluations. Un module sur l'\gls{ia} générative (généralités, conseils d'usage, éthique) est prévu pour la rentrée 2026-2027. L'approche privilégie la sobriété, la mutualisation des ressources internes, et l'inscription dans la dynamique collective de l'\gls{esr} français.

\bigskip

\section*{Abstract}

The European AI Act, which came into force in August 2024 and classifies education as a high-risk sector, requires higher education institutions to structure their approach to generative AI. This synthesis document, intended for the management of Polytech Annecy-Chambéry, analyzes the technical and scientific foundations of generative AI, examines the ethical and regulatory framework governing its deployment, documents the actual practices of actors in French higher education and research, and proposes an operational roadmap adapted to the constraints of an engineering school.

The methodology is based on a structured documentary analysis organized into seven thematic areas (foundations, ethics, pedagogical impacts, usage practices, governance, AI as discipline, Polytech specificities), drawing on over 90 institutional and scientific sources. Recent national surveys reveal massive adoption by students (75\% in Grandes Écoles), moderate adoption by faculty (52\%), and limited adoption by administrative staff, with specific barriers for each group: regulatory uncertainty (74\% of students do not declare their usage), lack of practical training (65\% of faculty), and employment concerns (72\% of administrative staff).

Seven priority recommendations structure the 2026-2028 roadmap: formal coordination with the USMB AI coordinator, internal survey of the three stakeholder groups, adoption of the USMB charter with sectoral adaptations, internal training mobilizing the MIAI research engineer and expert faculty, exploration of infrastructure options (ILaaS, school AI, Mistral partnership), identification of AI champions per department, and integration of the AIAS framework for assessments. A module on generative AI (fundamentals, usage guidelines, ethics) is planned for the 2026-2027 academic year. The approach prioritizes frugality, pooling of internal resources, and participation in the collective dynamics of French higher education and research.

\newpage