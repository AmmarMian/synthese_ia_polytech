\section{Méthodologie de recherche documentaire}
\label{sec:1_methodologie}

L'émergence de l'intelligence artificielle (\gls{ia}) générative dans l'enseignement supérieur depuis 2023 impose aux établissements une réflexion stratégique documentée. Ce travail s'appuie sur une recherche bibliographique structurée en sept axes thématiques, évaluant près de 90 sources selon une grille systématique. La méthodologie adoptée combine rigueur académique et transparence radicale, en assumant pleinement l'usage d'assistants d'\gls{ia} générative dans certaines phases du processus de recherche et de rédaction.

\subsection{Stratégie de recherche structurée}

La recherche documentaire s'organise autour de sept axes thématiques complémentaires couvrant les dimensions clés de l'intégration de l'\gls{ia} dans une école d'ingénieurs : les fondements techniques et théoriques (Axe 1), l'éthique et la réglementation (Axe 2), la pédagogie et l'apprentissage (Axe 3), les usages et pratiques des acteurs (Axe 4), la gouvernance institutionnelle (Axe 5), l'\gls{ia} dans les métiers de l'ingénierie (Axe 6), et les spécificités de Polytech Annecy-Chambéry (Axe 7). Cette architecture permet d'articuler une couverture exhaustive avec une cohérence analytique.

Chaque source a été évaluée selon quatre critères pondérés : l'autorité de l'auteur ou de l'institution (6 points), l'actualité dans un domaine en évolution rapide (3 points), la rigueur méthodologique (5 points), et la pertinence pour le contexte d'une école publique d'ingénieurs (4 points). Les sources totalisant 16 points ou plus ont fait l'objet d'une analyse approfondie ; celles en-deçà de 13 points n'ont pas été retenues. Cette grille d'évaluation systématique, détaillée dans le Tableau~\ref{tab:grille_eval}, assure la reproductibilité de la démarche.

\begin{table}[htbp]
\centering
\caption{Grille d'évaluation des sources documentaires}
\label{tab:grille_eval}
\begin{tabular}{lcp{8cm}}
\toprule
\textbf{Critère} & \textbf{Points} & \textbf{Description} \\
\midrule
Autorité & /6 & Crédibilité de l'auteur ou de l'institution, processus de validation (peer-review, comité scientifique) \\
Actualité & /3 & Pertinence temporelle dans un domaine en évolution rapide (priorité 2023-2025) \\
Rigueur & /5 & Méthodologie explicite, taille d'échantillon, reproductibilité des résultats \\
Pertinence & /4 & Alignement avec les objectifs du projet et le contexte d'une école d'ingénieurs \\
\midrule
\textbf{Total} & \textbf{/18} & \\
\bottomrule
\end{tabular}
\end{table}

Les seuils de décision ont été établis empiriquement après évaluation d'une première série de sources test. Un score de 16 à 18 identifie les sources majeures justifiant une analyse approfondie et une intégration centrale dans l'argumentaire. Les sources entre 13 et 15 points sont considérées comme utiles et consultées de manière ciblée. En-deçà de 13, les sources sont écartées pour des raisons de fiabilité, d'actualité insuffisante, ou de faible pertinence pour notre contexte.

\subsection{Constitution du corpus documentaire}

Le corpus repose sur cinq documents institutionnels identifiés pour leur autorité et leur exhaustivité : le rapport Pascal-Taddei remis au Ministère de l'Enseignement supérieur en juillet 2025~\cite{pascal2025ia}, le référentiel \gls{unesco} de compétences en \gls{ia} pour les enseignants~\cite{unesco2024competences}, la note de l'\gls{inria} sur l'\gls{ia} générative dans la recherche~\cite{inria2025note}, le cadre d'usage du Ministère de l'Éducation nationale~\cite{men2025cadre}, et la charte d'usage de l'Institut National du Service Public~\cite{insp2025charte}. Ils posent le cadre conceptuel, réglementaire et opérationnel de référence pour l'enseignement supérieur français.

La base initiale a été enrichie par une recherche ciblée privilégiant les méta-analyses récentes~\cite{wang2025meta,deng2025chatgpt,ma2025metaanalysis}, les enquêtes d'usages à large échantillon (N > 1000)~\cite{nikolic2025perceptions}, et les documents de gouvernance institutionnelle opérationnels. Une attention particulière a été accordée aux sources internationales de référence (\gls{unesco}, \gls{ocde}, Stanford HAI) et au cadre réglementaire européen~\cite{aiact2024,rgpd2016}. Le corpus final compte environ 90 sources, dont plus de 50 sont intégrées à la bibliographie.

Trois critères ont guidé cette sélection. L'actualité d'abord : dans un domaine en évolution rapide, priorité aux publications 2023-2025, avec quelques références antérieures pour les fondements conceptuels établis. La rigueur ensuite : préférence aux méta-analyses, revues systématiques et enquêtes à large échantillon plutôt qu'aux études de cas isolées. La pertinence enfin : chaque source devait contribuer au moins à l'un des sept axes thématiques dans le contexte spécifique d'une école d'ingénieurs.

Cette méthodologie présente des limites assumées. Le corpus privilégie les sources francophones et européennes, reflétant le contexte d'un établissement public français. Les publications les plus récentes (postérieures à novembre 2025) n'ont pu être intégrées, dans un domaine où l'actualité technologique évolue rapidement. Enfin, les preprints non validés et les sources secondaires sans garantie de rigueur ont été systématiquement exclus, au risque d'écarter certaines contributions innovantes mais encore non consolidées.

\subsection{Note méthodologique : usage de l'\gls{ia} générative dans ce travail}

Ce document a été rédigé dans le cadre d'une collaboration entre l'auteur humain et des assistants d'intelligence artificielle générative. Loin d'être une limite, cette approche illustre les pratiques que nous recommandons : une utilisation transparente, critique et responsable de ces outils. Les modèles utilisés sont Claude Opus 4.5 et Claude Sonnet 4 (Anthropic), accessibles via l'interface claude.ai et l'outil en ligne de commande Claude Code.

Le processus de recherche et de rédaction s'est déroulé en cinq étapes distinctes. La recherche bibliographique initiale a été réalisée intégralement par l'auteur humain, aboutissant à la collecte des six documents institutionnels de référence et à l'identification des axes thématiques structurants. Le développement méthodologique, notamment la grille d'évaluation des sources et la structuration en sept axes, a résulté d'une élaboration collaborative où l'\gls{ia} a proposé des cadres d'analyse affinés par l'expertise humaine. L'analyse documentaire a mobilisé Claude avec les fonctionnalités de recherche web et Deep Research pour compléter le corpus initial par des méta-analyses et sources internationales, portant le total à environ 90 sources. Les synthèses par axe thématique ont été produites par Claude puis systématiquement relues, corrigées et validées par l'auteur. Enfin, la rédaction du présent document combine génération assistée par Claude Code et révision humaine substantielle.

La répartition estimée entre rédaction humaine directe (40~\%) et génération assistée avec révision substantielle (60~\%) reflète un usage stratégique de l'\gls{ia} : déléguer la structuration et la synthèse documentaire, tout en conservant la responsabilité intellectuelle des analyses, des choix méthodologiques, et des recommandations. Toutes les références bibliographiques ont été vérifiées manuellement. Les données factuelles issues des synthèses ont été systématiquement contrôlées dans les documents sources primaires. Les recommandations finales engagent exclusivement l'auteur.

Cette transparence méthodologique vise à illustrer concrètement les principes que nous défendons dans ce rapport : assumer l'usage de l'\gls{ia}, documenter les processus, maintenir l'expertise et la responsabilité humaines. La méthodologie adoptée démontre qu'une intégration réfléchie de l'\gls{ia} générative dans les activités académiques peut accroître l'efficacité du travail intellectuel sans compromettre sa rigueur, à condition de respecter des principes clairs de transparence, de vérification, et de responsabilité.
