\section{Recommandations et feuille de route}
\label{sec:7_recommandations}

Les sections précédentes ont établi les fondements techniques et scientifiques de l'\gls{ia} générative, analysé le cadre éthique et réglementaire qui encadre son déploiement, documenté les pratiques réelles des acteurs de l'enseignement supérieur, et examiné les modèles de gouvernance émergents. Cette section traduit ces analyses en recommandations concrètes et opérationnelles, articulées selon une feuille de route réaliste pour Polytech Annecy-Chambéry. Le principe directeur demeure la sobriété, la mutualisation des ressources internes et l'inscription dans la dynamique collective de l'\gls{esr} français~\cite{pascal2025ia}.

\subsection{Synthèse des priorités}

Quatre axes structurent les recommandations : la gouvernance institutionnelle, garantissant la coordination avec l'\gls{usmb} et la cohérence stratégique ; la formation des personnels et des étudiants, s'appuyant sur les ressources internes mobilisables ; l'infrastructure technique, à explorer selon l'évolution de l'écosystème national ; et l'adaptation des modalités d'évaluation, privilégiant la transparence sur la détection automatisée. Le tableau~\ref{tab:synthese_recommandations} synthétise les huit recommandations prioritaires issues de l'analyse transversale.

\begin{table}[htbp]
\centering
\caption{Synthèse des recommandations transversales}
\label{tab:synthese_recommandations}
\begin{tabular}{clc}
\toprule
\textbf{N°} & \textbf{Recommandation} & \textbf{Axe} \\
\midrule
R1 & Coordination formelle avec référent \gls{usmb} (Léo Vanbervilet) & Gouvernance \\
R2 & Enquête interne (étudiants, enseignants, BIATSS) & Diagnostic \\
R3 & Adoption charte \gls{usmb} avec adaptations ingénieur & Gouvernance \\
R4 & Formation interne via IR MIAI et EC experts & Formation \\
R5 & Exploration infrastructure (ILaaS, IA école, Mistral) & Infrastructure \\
R6 & Identification champions \gls{ia} par département & Formation \\
R7 & Repenser évaluations (cadre \gls{aias}) & Évaluation \\
\bottomrule
\end{tabular}
\end{table}

\medskip

La coordination avec le référent \gls{ia} de l'\gls{usmb} constitue le préalable à toute action autonome (R1). L'enquête interne auprès des trois publics — étudiants, enseignants-chercheurs, personnels administratifs — permettra de documenter les usages réels et d'adapter les actions aux besoins spécifiques de Polytech (R2). L'adoption de la charte \gls{usmb}, une fois votée, sera complétée par des adaptations sectorielles couvrant les spécificités des écoles d'ingénieurs : projets industriels sous clauses de confidentialité, gestion du code source, propriété intellectuelle des livrables co-générés avec l'\gls{ia} (R3). La formation des personnels s'appuiera exclusivement sur les ressources internes : l'ingénieur de recherche financé par l'institut MIAI, les enseignants-chercheurs experts en \gls{ia} et apprentissage automatique, et les retours d'expérience des partenaires industriels (R4). L'infrastructure technique fera l'objet d'une exploration comparative des options disponibles — ILaaS, développement d'une \gls{ia} école, partenariat Amue-Mistral — sans privilégier une solution unique a priori (R5). L'identification de champions \gls{ia} au sein de chaque département permettra de structurer un accompagnement par les pairs, levier documenté comme particulièrement efficace avec des taux d'adoption rapides de 60 à 70\% en quelques mois (R6)~\cite{pascal2025ia,unesco2024competences}. Enfin, l'adoption du cadre \gls{aias} comme alternative aux détecteurs automatisés inefficaces permettra de structurer la transparence d'usage plutôt que la surveillance (R7)~\cite{perkins2024detection}.

\subsection{Feuille de route 2026-2028}

La temporalité d'action s'organise autour de trois horizons : court terme (premier semestre 2026), moyen terme (second semestre 2026 à 2027) et long terme (2027-2028). Le tableau~\ref{tab:feuille_route} détaille les actions prioritaires par période.

\begin{table}[htbp]
\centering
\caption{Feuille de route \gls{ia} Polytech Annecy-Chambéry 2026-2028}
\label{tab:feuille_route}
\begin{tabular}{lp{10cm}}
\toprule
\textbf{Horizon} & \textbf{Actions prioritaires} \\
\midrule
\textbf{S1 2026} &
(1) Coordination formelle avec référent \gls{usmb} \newline
(2) Lancement enquête interne (étudiants, enseignants, BIATSS) \newline
(3) Sensibilisation via ressources gratuites (MOOC AI4T, AMUE) \newline
(4) Exploration options infrastructure \\
\midrule
\textbf{S2 2026 - 2027} &
(1) Adoption charte \gls{usmb} avec adaptations ingénieur \newline
(2) Formation enseignants par IR MIAI et EC experts \newline
(3) Identification champions \gls{ia} par département \newline
(4) Repenser évaluations (intégration cadre \gls{aias}) \\
\midrule
\textbf{2027-2028} &
(1) Conformité \gls{aiact} systèmes haut risque (août 2027) \newline
(2) Partage d'expérience avec réseau Polytech \\
\bottomrule
\end{tabular}
\end{table}

\medskip

Le premier semestre 2026 concentre les actions structurantes. L'obligation d'AI Literacy instituée par l'article 4 de l'\gls{aiact} est effective depuis février 2025 : les actions de sensibilisation doivent désormais être documentées et tracées. L'enquête interne reprendra le questionnaire national déployé par le ministère auprès de 30~000 répondants, garantissant ainsi la comparabilité des résultats avec les données nationales. Les ressources de formation gratuites disponibles — MOOC FUN AI4T (2-3 heures), Class'Code (6 heures), webconférences AMUE, formations URFIST — constituent le socle de la sensibilisation de masse. L'exploration des options d'infrastructure technique se fera en parallèle, sans privilégier de choix définitif compte tenu de l'évolution rapide de l'écosystème (fédération ILaaS, partenariat Amue-Mistral en cours de déploiement, développement d'une solution autonome)~\cite{pascal2025ia,aiact2024}.

Le second semestre 2026 et l'année 2027 marquent la phase de consolidation institutionnelle. L'adoption de la charte \gls{usmb}, une fois votée par le conseil d'administration de l'université, sera complétée par une section dédiée aux spécificités des écoles d'ingénieurs : gestion des projets industriels comportant des clauses de non-divulgation, traçabilité de l'assistance \gls{ia} dans le développement de code source, clarification de la propriété intellectuelle des livrables co-générés, et articulation avec les exigences du référentiel \gls{cti}. La formation des enseignants-chercheurs et personnels administratifs sera animée par l'ingénieur de recherche MIAI et les enseignants-chercheurs experts, selon une approche privilégiant les cas d'usage concrets disciplinaires plutôt que les formations théoriques génériques. L'identification de champions \gls{ia} au sein de chaque département structurera un accompagnement par les pairs, modalité documentée comme particulièrement efficace pour l'appropriation des outils. La refonte des modalités d'évaluation intégrera le cadre \gls{aias} (5 niveaux de transparence), alternative aux détecteurs automatisés dont les taux de faux positifs sont documentés comme inacceptables. Un module sur l'\gls{ia} générative — généralités, conseils d'usage et éthique — est d'ores et déjà prévu pour la rentrée 2026-2027, s'inscrivant dans l'évolution du référentiel \gls{cti}~\cite{demoes2025charte,cge2024enquete,perkins2024detection}.

La période 2027-2028 concentre les actions de conformité réglementaire. L'échéance d'août 2027 pour la mise en conformité des systèmes à haut risque en éducation (évaluation, orientation, surveillance des examens) impose une préparation structurée dès 2026. Le partage d'expérience avec le réseau des écoles Polytech s'inscrira dans une logique de contribution collective plutôt que de leadership : l'objectif demeure l'inscription dans la dynamique nationale de l'\gls{esr} français~\cite{aiact2024,pascal2025ia}.

\subsection{Conditions de réussite}

La réussite de cette feuille de route suppose de lever les freins documentés dans les enquêtes nationales et de mobiliser les leviers d'action dont l'efficacité a été mesurée. Les freins varient fortement selon les publics. Chez les étudiants, l'incertitude sur les règles autorisées conduit 74\% d'entre eux à ne pas déclarer leur usage de l'\gls{ia}, générant une anxiété éthique et un flou réglementaire paralysant. Chez les enseignants-chercheurs, le manque de formation pratique constitue le frein majeur, invoqué par 65\% des non-utilisateurs, tandis que l'absence de politique claire au sein des établissements conduit 44\% d'entre eux à un attentisme institutionnel. Les personnels administratifs, techniques et bibliothèques expriment des inquiétudes massives : 72\% sont préoccupés par l'impact de l'\gls{ia} sur leur salaire, 80\% n'ont reçu aucune formation, et 44\% estiment que l'\gls{ia} ne peut pas aider leur travail, révélant des cas d'usage flous pour leurs métiers~\cite{pascal2025ia,cge2024enquete}.

Face à ces freins, les recherches convergent vers des leviers dont l'efficacité a été documentée. Les guidelines claires définissant précisément les usages autorisés et interdits constituent le levier le plus efficace pour réduire l'anxiété étudiante. La formation pratique axée sur des cas d'usage concrets disciplinaires s'avère supérieure aux formations théoriques génériques. Le peer learning via des champions \gls{ia} au sein des départements génère une adoption rapide, documentée à 60-70\% en quelques mois. La communication claire et rassurante sur l'emploi, expliquant le « pourquoi » de la transformation plutôt que l'imposant, réduit les résistances des personnels administratifs. Enfin, le soutien hiérarchique visible de la direction constitue un facteur critique : 43\% des initiatives échouent faute de sponsorship institutionnel~\cite{unesco2024competences,pascal2025ia}.

Les ressources mobilisables pour Polytech Annecy-Chambéry s'inscrivent dans une logique de sobriété et de mutualisation. L'ingénieur de recherche \gls{ia}, dont le poste est financé par l'institut MIAI de l'Université Grenoble Alpes et rattaché à la direction de Polytech, accompagnera la montée en compétences des personnels. Les enseignants-chercheurs disposant de compétences en \gls{ia} et apprentissage automatique dans les différentes filières seront mobilisés pour les formations internes. Les partenariats avec les industriels — via les stages, les projets et le réseau Alumni — permettront des retours d'expérience terrain sur les usages \gls{ia} en entreprise. Les ressources nationales gratuites (MOOC FUN AI4T et Class'Code, formations AMUE, réseau URFIST, référentiel Pix \gls{ia}) complèteront le dispositif sans investissement budgétaire. La coordination avec l'\gls{usmb} permettra de mutualiser l'infrastructure via la DSI universitaire et, pour les besoins de calcul recherche, de s'appuyer sur le mésocentre MUST (mutualisé CNRS/\gls{usmb})~\cite{unesco2024competences,pascal2025ia}.

Polytech Annecy-Chambéry s'inscrit ainsi dans la dynamique collective de l'enseignement supérieur et de la recherche français. L'enjeu n'est pas technologique mais organisationnel : accompagner le changement, structurer la coordination entre niveaux école et université, mobiliser les compétences existantes, rassurer les personnels. Les technologies \gls{ia} évoluent rapidement ; une clause de révision annuelle de la charte et des dispositifs s'impose donc comme condition de pérennité. Cette approche sobre et pragmatique permet d'avancer sans attendre des moyens budgétaires conséquents, tout en garantissant la conformité réglementaire et l'accompagnement des acteurs~\cite{pascal2025ia,cge2024enquete}.

\bigskip

La mise en œuvre de cette feuille de route constitue une première étape structurante. Au-delà de ces recommandations opérationnelles, quelles perspectives s'ouvrent pour l'enseignement supérieur français à l'ère de l'\gls{ia} générative ? La conclusion du document examine ces enjeux prospectifs et les questions ouvertes qui demeurent.
