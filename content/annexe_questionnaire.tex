\appendix

\section{Questionnaire : Usages de l'\gls{ia} générative à Polytech}
\label{annexe:questionnaire}

\subsection*{Note méthodologique}

Ce questionnaire vise à documenter les usages réels de l'\gls{ia} générative au sein de Polytech Annecy-Chambéry, afin d'adapter les recommandations institutionnelles aux besoins spécifiques de chaque public (étudiants, enseignants-chercheurs, personnels administratifs et techniques). Quatre objectifs structurent cette enquête : (1) comparer les pratiques locales aux données nationales (n=30~000 répondants de l'enquête ministérielle), (2) identifier les besoins d'accompagnement différenciés par public, (3) repérer les champions \gls{ia} potentiels au sein des départements, et (4) documenter les enjeux spécifiques aux écoles d'ingénieurs (confidentialité des données industrielles, usage du code assisté par \gls{ia}, spécificités disciplinaires).

La méthodologie repose sur trois piliers validés scientifiquement. Les items du questionnaire national sont repris intégralement pour garantir la comparabilité des résultats (sections A à E, tronc commun). Les échelles psychométriques validées internationalement sont intégrées : échelle d'usage académique de ChatGPT (Abbas et al., 2024, 8 items, $\alpha>0.70$), modèle UTAUT2 adapté à l'\gls{ia} (5 construits), et échelle d'auto-efficacité GSE-6AI (6 items). Enfin, des questions spécifiques aux écoles d'ingénieurs complètent le dispositif (section F, différenciée par public)~\cite{pascal2025ia,cge2024enquete,abbas2024harmful}.

Le questionnaire adopte une architecture modulaire : un tronc commun de 42 items (sections A à E) pour tous les répondants, suivi d'une branche spécifique de 8 items selon le profil déclaré (section F), et d'une clôture commune (section G). La durée estimée de réponse est de 10 à 13 minutes. L'anonymat complet est garanti : aucune donnée personnelle identifiante n'est collectée. Les données agrégées serviront exclusivement à l'élaboration du plan d'action institutionnel. Un engagement de restitution des résultats aux participants est pris.

\bigskip

\subsection{Section A — Profil et contexte}

\textbf{A1.} Vous êtes :
\begin{itemize}[label=$\square$, leftmargin=1.5cm]
  \item Étudiant·e
  \item Enseignant·e-chercheur·euse
  \item Personnel administratif, technique, bibliothèque, ingénierie, social et santé (BIATSS)
\end{itemize}

\textbf{A2.} \textit{[Si Étudiant]} Dans quelle filière êtes-vous inscrit·e ?
\begin{itemize}[label=$\square$, leftmargin=1.5cm]
  \item Informatique et Microélectronique des Systèmes Intelligents (IMSI)
  \item Matériaux (MAT)
  \item Mécanique (MEC)
  \item Instrumentation, Automatique et Informatique (IAI)
  \item Environnement, Bâtiment, Énergie (EBE)
  \item Génie Biomédical (GBM)
\end{itemize}

\textbf{A3.} \textit{[Si Étudiant]} Vous êtes en :
\begin{itemize}[label=$\square$, leftmargin=1.5cm]
  \item 3\textsuperscript{e} année (1\textsuperscript{re} année cycle ingénieur)
  \item 4\textsuperscript{e} année (2\textsuperscript{e} année cycle ingénieur)
  \item 5\textsuperscript{e} année (3\textsuperscript{e} année cycle ingénieur)
\end{itemize}

\textbf{A4.} \textit{[Si Enseignant-chercheur]} Quelle est votre discipline principale d'enseignement ?
\begin{itemize}[label=$\square$, leftmargin=1.5cm]
  \item Informatique et sciences du numérique
  \item Mathématiques et modélisation
  \item Physique et sciences de l'ingénieur
  \item Mécanique et matériaux
  \item Environnement et énergétique
  \item Sciences humaines et sociales, langues
  \item Autre : \underline{\hspace{4cm}}
\end{itemize}

\textbf{A5.} \textit{[Si BIATSS]} Quelle est votre fonction principale ?
\begin{itemize}[label=$\square$, leftmargin=1.5cm]
  \item Administration (gestion, finances, RH)
  \item Scolarité et relations étudiantes
  \item Informatique et systèmes d'information
  \item Ingénierie pédagogique
  \item Bibliothèque et documentation
  \item Technique et logistique
  \item Autre : \underline{\hspace{4cm}}
\end{itemize}

\textbf{A6.} Connaissez-vous l'existence de règles ou de recommandations sur l'usage de l'\gls{ia} générative au sein de Polytech ou de l'\gls{usmb} ?
\begin{itemize}[label=$\square$, leftmargin=1.5cm]
  \item Oui, je connais les règles
  \item Non, je ne connais pas les règles
  \item Je ne sais pas s'il existe des règles
\end{itemize}

\medskip

\subsection{Section B — Usage actuel de l'\gls{ia} générative}

\textbf{B1.} Utilisez-vous des outils d'\gls{ia} générative (ChatGPT, Claude, Gemini, Copilot, etc.) ?
\begin{itemize}[label=$\square$, leftmargin=1.5cm]
  \item Oui
  \item Non $\rightarrow$ \textit{Passez à la question B9}
\end{itemize}

\textbf{B2.} Depuis combien de temps utilisez-vous l'\gls{ia} générative ?
\begin{itemize}[label=$\square$, leftmargin=1.5cm]
  \item Moins de 3 mois
  \item Entre 3 et 6 mois
  \item Entre 6 mois et 1 an
  \item Entre 1 an et 2 ans
  \item Plus de 2 ans
\end{itemize}

\textbf{B3.} À quelle fréquence utilisez-vous l'\gls{ia} générative ?
\begin{itemize}[label=$\square$, leftmargin=1.5cm]
  \item Jamais (je ne l'utilise plus)
  \item Rarement (moins d'une fois par mois)
  \item Occasionnellement (1 à 3 fois par mois)
  \item Régulièrement (1 à 2 fois par semaine)
  \item Fréquemment (plusieurs fois par semaine)
  \item Quotidiennement
\end{itemize}

\textbf{B4.} Dans quel(s) contexte(s) utilisez-vous l'\gls{ia} générative ? \textit{(Plusieurs réponses possibles)}
\begin{itemize}[label=$\square$, leftmargin=1.5cm]
  \item \textit{[Étudiant]} Pour mes études (travaux, projets, révisions)
  \item \textit{[Enseignant]} Pour préparer mes cours et mes supports pédagogiques
  \item \textit{[Enseignant]} Pour mes activités de recherche
  \item \textit{[BIATSS]} Pour mes tâches professionnelles administratives
  \item Usage personnel (hors cadre professionnel/études)
\end{itemize}

\textbf{B5.} Quel est votre niveau de compétence auto-évalué dans l'utilisation de l'\gls{ia} générative ?
\begin{itemize}[label=$\square$, leftmargin=1.5cm]
  \item Débutant (je découvre ces outils)
  \item Intermédiaire (j'utilise les fonctions de base)
  \item Avancé (je maîtrise l'ingénierie de prompts et les usages complexes)
  \item Expert (je forme d'autres personnes à leur usage)
\end{itemize}

\textbf{B6.} Avez-vous déjà déclaré auprès d'un enseignant votre usage de l'\gls{ia} générative pour un travail académique ?
\begin{itemize}[label=$\square$, leftmargin=1.5cm]
  \item Oui, toujours
  \item Oui, parfois
  \item Non, jamais
  \item Non concerné (je ne l'utilise pas pour les travaux académiques)
\end{itemize}

\textbf{B7.} Avez-vous reçu une formation à l'usage de l'\gls{ia} générative ?
\begin{itemize}[label=$\square$, leftmargin=1.5cm]
  \item Oui, une formation institutionnelle (Polytech, \gls{usmb})
  \item Oui, une formation externe
  \item Non, je suis autodidacte
  \item Non, je n'ai reçu aucune formation
\end{itemize}

\textbf{B8.} Seriez-vous intéressé·e par une formation sur l'usage éthique et efficace de l'\gls{ia} générative ?
\begin{itemize}[label=$\square$, leftmargin=1.5cm]
  \item Oui, très intéressé·e
  \item Oui, plutôt intéressé·e
  \item Non, peu intéressé·e
  \item Non, pas du tout intéressé·e
\end{itemize}

\textbf{B9.} \textit{[Si non-utilisateur]} Quelle est la principale raison pour laquelle vous n'utilisez pas l'\gls{ia} générative ? \textit{(Une seule réponse)}
\begin{itemize}[label=$\square$, leftmargin=1.5cm]
  \item Je ne connais pas ces outils
  \item Je ne vois pas d'utilité pour mon travail/mes études
  \item Je ne sais pas comment les utiliser
  \item Je crains les problèmes d'intégrité académique ou éthique
  \item Je ne fais pas confiance à la fiabilité des résultats
  \item L'établissement n'a pas donné de règles claires
  \item Autre raison : \underline{\hspace{4cm}}
\end{itemize}

\medskip

\subsection{Section C — Outils et tâches}

\textbf{C1.} Quels outils d'\gls{ia} générative utilisez-vous ? \textit{(Plusieurs réponses possibles)}
\begin{itemize}[label=$\square$, leftmargin=1.5cm]
  \item ChatGPT (OpenAI) — version gratuite
  \item ChatGPT (OpenAI) — version payante (Plus, Team, Enterprise)
  \item Claude (Anthropic)
  \item Google Gemini
  \item Microsoft Copilot (Bing Chat)
  \item GitHub Copilot
  \item DeepL (traduction)
  \item Mistral AI
  \item Perplexity
  \item Autre : \underline{\hspace{4cm}}
\end{itemize}

\textbf{C2.} Pour quelles tâches utilisez-vous l'\gls{ia} générative ? \textit{(Plusieurs réponses possibles)}
\begin{itemize}[label=$\square$, leftmargin=1.5cm]
  \item Recherche d'information et documentation
  \item Rédaction de textes (rapports, courriels, synthèses)
  \item Résumé de documents longs
  \item Traduction de textes
  \item Génération ou complétion de code informatique
  \item Débogage de code
  \item Clarification de concepts théoriques
  \item Génération d'exercices ou de questions d'examen
  \item Création de supports visuels (diagrammes, schémas)
  \item Aide à la correction ou à l'évaluation de travaux
  \item Autre : \underline{\hspace{4cm}}
\end{itemize}

\textbf{C3.} Quelle proportion de votre production finale est générée directement par l'\gls{ia} ?
\begin{itemize}[label=$\square$, leftmargin=1.5cm]
  \item 0\% (je n'utilise l'\gls{ia} que pour des idées, pas pour la rédaction finale)
  \item Moins de 25\%
  \item Entre 25\% et 50\%
  \item Entre 50\% et 75\%
  \item Plus de 75\%
\end{itemize}

\textbf{C4.} Vérifiez-vous systématiquement les informations fournies par l'\gls{ia} générative ?
\begin{itemize}[label=$\square$, leftmargin=1.5cm]
  \item Toujours
  \item Souvent
  \item Parfois
  \item Rarement
  \item Jamais
\end{itemize}

\medskip

\subsection{Section D — Perceptions et attitudes}

Les affirmations suivantes utilisent une échelle de Likert en 5 points : 1 = Pas du tout d'accord ; 2 = Plutôt pas d'accord ; 3 = Neutre ; 4 = Plutôt d'accord ; 5 = Tout à fait d'accord.

\textbf{D1.} L'utilisation de l'\gls{ia} générative améliore ma productivité.
\begin{itemize}[label=$\square$, leftmargin=1.5cm]
  \item 1 \quad $\square$ 2 \quad $\square$ 3 \quad $\square$ 4 \quad $\square$ 5
\end{itemize}

\textbf{D2.} Apprendre à utiliser l'\gls{ia} générative est facile pour moi.
\begin{itemize}[label=$\square$, leftmargin=1.5cm]
  \item 1 \quad $\square$ 2 \quad $\square$ 3 \quad $\square$ 4 \quad $\square$ 5
\end{itemize}

\textbf{D3.} Je fais confiance aux informations fournies par l'\gls{ia} générative.
\begin{itemize}[label=$\square$, leftmargin=1.5cm]
  \item 1 \quad $\square$ 2 \quad $\square$ 3 \quad $\square$ 4 \quad $\square$ 5
\end{itemize}

\textbf{D4.} Mes enseignants/collègues encouragent l'utilisation de l'\gls{ia} générative.
\begin{itemize}[label=$\square$, leftmargin=1.5cm]
  \item 1 \quad $\square$ 2 \quad $\square$ 3 \quad $\square$ 4 \quad $\square$ 5
\end{itemize}

\textbf{D5.} J'ai l'intention d'utiliser davantage l'\gls{ia} générative à l'avenir.
\begin{itemize}[label=$\square$, leftmargin=1.5cm]
  \item 1 \quad $\square$ 2 \quad $\square$ 3 \quad $\square$ 4 \quad $\square$ 5
\end{itemize}

\textbf{D6.} Je suis confiant·e dans ma capacité à utiliser l'\gls{ia} générative efficacement.
\begin{itemize}[label=$\square$, leftmargin=1.5cm]
  \item 1 \quad $\square$ 2 \quad $\square$ 3 \quad $\square$ 4 \quad $\square$ 5
\end{itemize}

\textbf{D7.} Je sais comment vérifier la fiabilité des réponses de l'\gls{ia} générative.
\begin{itemize}[label=$\square$, leftmargin=1.5cm]
  \item 1 \quad $\square$ 2 \quad $\square$ 3 \quad $\square$ 4 \quad $\square$ 5
\end{itemize}

\textbf{D8.} L'\gls{ia} générative représente une opportunité pour mon travail/mes études.
\begin{itemize}[label=$\square$, leftmargin=1.5cm]
  \item 1 \quad $\square$ 2 \quad $\square$ 3 \quad $\square$ 4 \quad $\square$ 5
\end{itemize}

\textbf{D9.} L'\gls{ia} générative représente une menace pour mon travail/mes études.
\begin{itemize}[label=$\square$, leftmargin=1.5cm]
  \item 1 \quad $\square$ 2 \quad $\square$ 3 \quad $\square$ 4 \quad $\square$ 5
\end{itemize}

\textbf{D10.} Je suis préoccupé·e par les inexactitudes (hallucinations) de l'\gls{ia} générative.
\begin{itemize}[label=$\square$, leftmargin=1.5cm]
  \item 1 \quad $\square$ 2 \quad $\square$ 3 \quad $\square$ 4 \quad $\square$ 5
\end{itemize}

\medskip

\subsection{Section E — Freins et besoins}

\textbf{E1.} Quels sont les principaux freins à votre utilisation de l'\gls{ia} générative ? \textit{(Plusieurs réponses possibles)}
\begin{itemize}[label=$\square$, leftmargin=1.5cm]
  \item Manque de formation
  \item Incertitude sur les règles d'usage autorisées
  \item Crainte de problèmes d'intégrité académique ou déontologique
  \item Manque de confiance dans la fiabilité des résultats
  \item Préoccupations sur la confidentialité des données
  \item Temps nécessaire pour apprendre à utiliser ces outils
  \item Absence de soutien de ma hiérarchie ou de mes enseignants
  \item Coût des versions payantes
  \item Impact environnemental
  \item Autre : \underline{\hspace{4cm}}
\end{itemize}

\textbf{E2.} Quels types de formation souhaiteriez-vous recevoir sur l'\gls{ia} générative ? \textit{(Plusieurs réponses possibles)}
\begin{itemize}[label=$\square$, leftmargin=1.5cm]
  \item Introduction générale aux outils disponibles
  \item Ingénierie de prompts (formulation de requêtes efficaces)
  \item Usages éthiques et respect de l'intégrité académique
  \item Vérification de la fiabilité des résultats
  \item Applications spécifiques à ma discipline/mon métier
  \item Sécurité et confidentialité des données
  \item Impact environnemental et sobriété numérique
  \item Autre : \underline{\hspace{4cm}}
\end{itemize}

\textbf{E3.} Quelle modalité de formation préféreriez-vous ?
\begin{itemize}[label=$\square$, leftmargin=1.5cm]
  \item Formation en ligne asynchrone (MOOC, vidéos)
  \item Ateliers en présentiel (2-4 heures)
  \item Formation par les pairs (collègues/étudiants référents)
  \item Ressources documentaires (guides, fiches pratiques)
  \item Webinaires interactifs
  \item Autre : \underline{\hspace{4cm}}
\end{itemize}

\medskip

\subsection{Section F — Questions spécifiques}

\subsubsection{Version Étudiants}

\textbf{F1.} Utilisez-vous GitHub Copilot ou un assistant de code similaire pour vos projets informatiques ?
\begin{itemize}[label=$\square$, leftmargin=1.5cm]
  \item Oui, régulièrement
  \item Oui, occasionnellement
  \item Non, mais je connais cet outil
  \item Non, je ne connais pas cet outil
  \item Non concerné (je ne code pas)
\end{itemize}

\textbf{F2.} \textit{[Si utilisateur de code assisté]} Vérifiez-vous systématiquement le code généré par l'\gls{ia} avant de l'utiliser ?
\begin{itemize}[label=$\square$, leftmargin=1.5cm]
  \item Toujours
  \item Souvent
  \item Parfois
  \item Rarement
  \item Jamais
\end{itemize}

\textbf{F3.} Avez-vous déjà uploadé des données issues de projets industriels ou de stages dans un outil d'\gls{ia} générative public (ChatGPT, Claude, etc.) ?
\begin{itemize}[label=$\square$, leftmargin=1.5cm]
  \item Oui
  \item Non
  \item Je ne sais pas
  \item Non concerné (je n'ai pas encore de projet industriel ou stage)
\end{itemize}

\textbf{F4.} Êtes-vous conscient·e des risques liés à la confidentialité lorsque vous utilisez des outils d'\gls{ia} publics pour des données de projets industriels ?
\begin{itemize}[label=$\square$, leftmargin=1.5cm]
  \item Oui, tout à fait
  \item Oui, plutôt
  \item Non, plutôt pas
  \item Non, pas du tout
\end{itemize}

\textbf{F5.} Considérez-vous que l'usage de l'\gls{ia} générative pour rédiger un devoir est une forme de triche ?
\begin{itemize}[label=$\square$, leftmargin=1.5cm]
  \item Oui, dans tous les cas
  \item Oui, si l'enseignant ne l'a pas autorisé
  \item Non, si on déclare son usage
  \item Non, jamais
  \item Je ne sais pas
\end{itemize}

\textbf{F6.} Avez-vous déjà ressenti de l'anxiété ou de la culpabilité à utiliser l'\gls{ia} générative pour vos études ?
\begin{itemize}[label=$\square$, leftmargin=1.5cm]
  \item Oui, souvent
  \item Oui, parfois
  \item Non, rarement
  \item Non, jamais
\end{itemize}

\subsubsection{Version Enseignants-chercheurs}

\textbf{F1.} Utilisez-vous l'\gls{ia} générative pour préparer vos enseignements ?
\begin{itemize}[label=$\square$, leftmargin=1.5cm]
  \item Oui, régulièrement
  \item Oui, occasionnellement
  \item Non, jamais
\end{itemize}

\textbf{F2.} \textit{[Si oui]} Pour quelles tâches pédagogiques ? \textit{(Plusieurs réponses possibles)}
\begin{itemize}[label=$\square$, leftmargin=1.5cm]
  \item Préparation de supports de cours
  \item Création d'exercices ou de questions d'examen
  \item Génération d'exemples ou de cas d'étude
  \item Aide à la correction de travaux étudiants
  \item Traduction de ressources pédagogiques
  \item Autre : \underline{\hspace{4cm}}
\end{itemize}

\textbf{F3.} Utilisez-vous l'\gls{ia} générative dans vos activités de recherche ?
\begin{itemize}[label=$\square$, leftmargin=1.5cm]
  \item Oui, régulièrement
  \item Oui, occasionnellement
  \item Non, jamais
\end{itemize}

\textbf{F4.} \textit{[Si oui]} Pour quelles activités de recherche ? \textit{(Plusieurs réponses possibles)}
\begin{itemize}[label=$\square$, leftmargin=1.5cm]
  \item Revue de littérature et veille scientifique
  \item Rédaction d'articles ou de rapports
  \item Génération ou débogage de code de traitement de données
  \item Traduction d'articles
  \item Aide à la formulation d'hypothèses
  \item Autre : \underline{\hspace{4cm}}
\end{itemize}

\textbf{F5.} Estimez-vous que vos étudiants utilisent l'\gls{ia} générative pour leurs travaux ?
\begin{itemize}[label=$\square$, leftmargin=1.5cm]
  \item Oui, la majorité (plus de 70\%)
  \item Oui, environ la moitié (40-70\%)
  \item Oui, une minorité (10-40\%)
  \item Non, très peu (moins de 10\%)
  \item Je ne sais pas
\end{itemize}

\textbf{F6.} Avez-vous adapté vos modalités d'évaluation en raison de l'\gls{ia} générative ?
\begin{itemize}[label=$\square$, leftmargin=1.5cm]
  \item Oui, substantiellement
  \item Oui, légèrement
  \item Non, pas encore
  \item Non, je ne pense pas que ce soit nécessaire
\end{itemize}

\textbf{F7.} Seriez-vous prêt·e à devenir référent·e \gls{ia} pour accompagner vos collègues ?
\begin{itemize}[label=$\square$, leftmargin=1.5cm]
  \item Oui, volontiers
  \item Oui, si je reçois une formation préalable
  \item Non, je ne me sens pas compétent·e
  \item Non, je n'ai pas le temps
\end{itemize}

\subsubsection{Version BIATSS}

\textbf{F1.} Pour quelles tâches professionnelles l'\gls{ia} générative pourrait-elle vous aider ? \textit{(Plusieurs réponses possibles)}
\begin{itemize}[label=$\square$, leftmargin=1.5cm]
  \item Rédaction de courriels, comptes rendus, rapports
  \item Traitement et analyse de données
  \item Organisation et planification
  \item Recherche documentaire
  \item Traduction de documents
  \item Support utilisateurs (réponses aux questions récurrentes)
  \item Je ne vois pas d'application à mon travail
  \item Autre : \underline{\hspace{4cm}}
\end{itemize}

\textbf{F2.} Votre responsable hiérarchique vous encourage-t-il à utiliser l'\gls{ia} générative ?
\begin{itemize}[label=$\square$, leftmargin=1.5cm]
  \item Oui, activement
  \item Oui, mais sans accompagnement
  \item Non, ce sujet n'a jamais été évoqué
  \item Non, il/elle y est opposé·e
\end{itemize}

\textbf{F3.} Avez-vous accès à des outils d'\gls{ia} générative fournis par l'établissement ?
\begin{itemize}[label=$\square$, leftmargin=1.5cm]
  \item Oui
  \item Non
  \item Je ne sais pas
\end{itemize}

\textbf{F4.} Êtes-vous inquiet·ète de l'impact de l'\gls{ia} sur votre emploi ou votre rémunération ?
\begin{itemize}[label=$\square$, leftmargin=1.5cm]
  \item Oui, très inquiet·ète
  \item Oui, plutôt inquiet·ète
  \item Non, plutôt pas inquiet·ète
  \item Non, pas du tout inquiet·ète
\end{itemize}

\textbf{F5.} Pensez-vous que l'\gls{ia} générative pourrait améliorer votre qualité de vie au travail (gain de temps, réduction des tâches répétitives) ?
\begin{itemize}[label=$\square$, leftmargin=1.5cm]
  \item Oui, certainement
  \item Oui, probablement
  \item Non, probablement pas
  \item Non, certainement pas
\end{itemize}

\textbf{F6.} Avez-vous besoin d'un accompagnement managérial pour intégrer l'\gls{ia} dans vos pratiques professionnelles ?
\begin{itemize}[label=$\square$, leftmargin=1.5cm]
  \item Oui, absolument
  \item Oui, plutôt
  \item Non, plutôt pas
  \item Non, pas du tout
\end{itemize}

\medskip

\subsection{Section G — Clôture}

\textbf{G1.} Quelles actions prioritaires Polytech devrait-elle mener concernant l'\gls{ia} générative ? \textit{(Plusieurs réponses possibles)}
\begin{itemize}[label=$\square$, leftmargin=1.5cm]
  \item Définir des règles claires d'usage
  \item Proposer des formations pour tous les personnels
  \item Fournir des outils institutionnels sécurisés
  \item Créer un réseau de référents \gls{ia} par département
  \item Adapter les modalités d'évaluation des étudiants
  \item Sensibiliser aux enjeux éthiques et environnementaux
  \item Autre : \underline{\hspace{4cm}}
\end{itemize}

\textbf{G2.} Avez-vous des suggestions ou des commentaires complémentaires sur l'usage de l'\gls{ia} générative à Polytech ?

\vspace{3cm}

\textbf{G3.} Accepteriez-vous d'être contacté·e pour un entretien complémentaire ou pour participer à un groupe de travail sur l'\gls{ia} ?
\begin{itemize}[label=$\square$, leftmargin=1.5cm]
  \item Oui $\rightarrow$ Votre adresse courriel : \underline{\hspace{5cm}}
  \item Non
\end{itemize}

\bigskip

\noindent\textit{Merci pour votre participation. Vos réponses contribueront à adapter l'accompagnement institutionnel aux besoins réels de la communauté Polytech.}
